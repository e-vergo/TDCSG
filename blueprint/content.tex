\maketitle

\begin{abstract}
This blueprint documents the formal verification in Lean 4 of the critical radius theorem for two-disk compound symmetry groups from arXiv:2302.12950v1. The main result proves that the compound symmetry group GG₅ is infinite at the critical radius $r_c = \sqrt{3 + \phi}$, where $\phi = \frac{1+\sqrt{5}}{2}$ is the golden ratio.
\end{abstract}

\section{Main Theorem}

\begin{theorem}[GG5 Infinite at Critical Radius]\label{thm:main}
\lean{GG5_infinite_at_critical_radius}
\leanok
The compound symmetry group GG₅ is infinite at the critical radius $r_c = \sqrt{3 + \phi}$. Formally, there exists a point $x$ whose orbit under the group has arbitrarily large finite subsets.
\uses{thm:iet_infinite_orbit, def:gg5_critical, def:r_crit}
\end{theorem}

\section{Foundations}

\subsection{Critical Radius}

\begin{definition}[Critical Radius]\label{def:r_crit}
\lean{r_crit}
\leanok
The critical radius is defined as $r_c = \sqrt{3 + \phi}$ where $\phi$ is the golden ratio.
\end{definition}

\begin{lemma}[Critical Radius Positivity]\label{lem:r_crit_pos}
\lean{r_crit_pos}
\leanok
The critical radius is positive: $r_c > 0$.
\uses{def:r_crit}
\end{lemma}

\begin{lemma}[Critical Radius Approximation]\label{lem:r_crit_approx}
\lean{r_crit_approx}
\leanok
The critical radius satisfies $2.148 < r_c < 2.150$.
\uses{def:r_crit}
\end{lemma}

\begin{lemma}[Critical Radius Minimal Polynomial]\label{lem:r_crit_minimal_poly}
\lean{r_crit_minimal_poly}
\leanok
The critical radius satisfies the minimal polynomial equation: $r_c^4 - 7r_c^2 + 11 = 0$.
\uses{def:r_crit}
\end{lemma}

\subsection{Fifth Root of Unity}

\begin{definition}[Fifth Root of Unity]\label{def:zeta5}
\lean{ζ₅}
\leanok
Let $\zeta_5 = e^{2\pi i/5}$ be the primitive fifth root of unity.
\end{definition}

\begin{lemma}[Fifth Power]\label{lem:zeta5_pow_five}
\lean{zeta5_pow_five}
\leanok
$\zeta_5^5 = 1$.
\uses{def:zeta5}
\end{lemma}

\begin{lemma}[Primitive Root]\label{lem:zeta5_primitive}
\lean{zeta5_isPrimitiveRoot}
\leanok
$\zeta_5$ is a primitive fifth root of unity.
\uses{def:zeta5, lem:zeta5_pow_five}
\end{lemma}

\begin{lemma}[Unit Norm]\label{lem:zeta5_abs}
\lean{zeta5_abs}
\leanok
$|\zeta_5| = 1$.
\uses{def:zeta5}
\end{lemma}

\begin{lemma}[Cyclotomic Sum]\label{lem:cyclotomic5_sum}
\lean{cyclotomic5_sum}
\leanok
$1 + \zeta_5 + \zeta_5^2 + \zeta_5^3 + \zeta_5^4 = 0$.
\uses{def:zeta5, lem:zeta5_primitive}
\end{lemma}

\begin{lemma}[Cosine Identity]\label{lem:cos_two_pi_fifth}
\lean{cos_two_pi_fifth}
\leanok
$\cos(2\pi/5) = (\phi - 1)/2$.
\uses{def:zeta5}
\end{lemma}

\subsection{Geometric Points}

\begin{definition}[Point E]\label{def:E}
\lean{E}
\leanok
The point $E$ is defined via the fifth root of unity and lies on the boundary of the left disk.
\uses{def:zeta5, def:r_crit}
\end{definition}

\begin{definition}[Point E']\label{def:Eprime}
\lean{E'}
\leanok
The point $E'$ is the reflection of $E$ and lies on the boundary of the right disk.
\uses{def:E}
\end{definition}

\begin{definition}[Point F]\label{def:F}
\lean{F}
\leanok
The point $F$ lies on the segment from $E'$ to $E$.
\uses{def:E, def:Eprime, def:zeta5}
\end{definition}

\begin{definition}[Point G]\label{def:G}
\lean{G}
\leanok
The point $G$ lies on the segment from $E'$ to $E$.
\uses{def:E, def:Eprime, def:zeta5}
\end{definition}

\begin{lemma}[E on Left Disk Boundary]\label{lem:E_on_left_disk}
\lean{E_on_left_disk_boundary}
\leanok
$\|E + 1\| = r_c$.
\uses{def:E, def:r_crit}
\end{lemma}

\begin{lemma}[E in Right Disk]\label{lem:E_in_right_disk}
\lean{E_in_right_disk}
\leanok
$\|E - 1\| \leq r_c$.
\uses{def:E, def:r_crit}
\end{lemma}

\begin{lemma}[E' on Right Disk Boundary]\label{lem:Eprime_on_right_disk}
\lean{E'_on_right_disk_boundary}
\leanok
$\|E' - 1\| = r_c$.
\uses{def:Eprime, def:r_crit}
\end{lemma}

\begin{lemma}[E' in Left Disk]\label{lem:Eprime_in_left_disk}
\lean{E'_in_left_disk}
\leanok
$\|E' + 1\| \leq r_c$.
\uses{def:Eprime, def:r_crit}
\end{lemma}

\subsection{Segment Structure}

\begin{lemma}[F on Segment]\label{lem:F_on_segment}
\lean{F_on_segment_E'E}
\leanok
$F$ lies on the segment from $E'$ to $E$: there exists $t \in [0,1]$ with $F = E' + t(E - E')$.
\uses{def:F, def:E, def:Eprime}
\end{lemma}

\begin{lemma}[G on Segment]\label{lem:G_on_segment}
\lean{G_on_segment_E'E}
\leanok
$G$ lies on the segment from $E'$ to $E$: there exists $t \in [0,1]$ with $G = E' + t(E - E')$.
\uses{def:G, def:E, def:Eprime}
\end{lemma}

\begin{lemma}[Segment Ordering]\label{lem:segment_ordering}
\lean{segment_ordering}
\leanok
The points are ordered as $E'$, $G$, $F$, $E$ with $0 < t_G < t_F < 1$.
\uses{def:E, def:Eprime, def:F, def:G}
\end{lemma}

\begin{definition}[Translation Lengths]\label{def:translation_lengths}
\lean{translation_length_1, translation_length_2}
\leanok
The translation lengths of the three segment maps.
\uses{def:E, def:Eprime, def:F, def:G}
\end{definition}

\begin{definition}[Segment Length]\label{def:segment_length}
\lean{segment_length}
\leanok
The length of the segment from $E'$ to $E$.
\uses{def:E, def:Eprime}
\end{definition}

\begin{theorem}[Segment Ratio is Golden]\label{thm:segment_ratio_golden}
\lean{segment_ratio_is_golden}
\leanok
The ratio of segment length to the first translation length equals the golden ratio:
$\frac{\text{segment\_length}}{\text{translation\_length}_1} = \phi$.
\uses{def:segment_length, def:translation_lengths}
\end{theorem}

\begin{theorem}[Translations Irrational]\label{thm:translations_irrational}
\lean{translations_irrational}
\leanok
The translation lengths are irrationally related: for any integers $q, r$ not both zero,
$q \cdot \text{translation\_length}_1 + r \cdot \text{translation\_length}_2 \neq 0$.
\uses{def:translation_lengths, thm:segment_ratio_golden}
\end{theorem}

\subsection{Two-Disk System}

\begin{definition}[Two-Disk System]\label{def:twodisk}
\lean{CompoundSymmetry.TwoDiskSystem}
\leanok
A two-disk system consists of two disks with specified radii and generators that are rotations.
\end{definition}

\begin{definition}[Generators]\label{def:generators}
\lean{genA, genB}
\leanok
The generators are rotations by $2\pi/5$ about the disk centers.
\uses{def:twodisk, def:zeta5}
\end{definition}

\begin{definition}[GG5 Critical System]\label{def:gg5_critical}
\lean{GG5_critical}
\leanok
The two-disk system GG₅ at the critical radius.
\uses{def:twodisk, def:r_crit, def:generators}
\end{definition}

\section{Generators and Isometries}

\begin{axiom}[Lens Intersection Preservation]\label{ax:lens_preserved}
\lean{lens_intersection_preserved_by_rotation}
If $z$ is in the intersection of both disks, then its rotation is also in the intersection.
\uses{def:r_crit, def:zeta5}
\end{axiom}

\begin{lemma}[GenA Preserves Intersection]\label{lem:genA_preserves}
\lean{genA_preserves_intersection}
\leanok
The generator $A$ preserves the disk intersection.
\uses{def:generators, ax:lens_preserved}
\end{lemma}

\begin{lemma}[GenB Preserves Intersection]\label{lem:genB_preserves}
\lean{genB_preserves_intersection}
\leanok
The generator $B$ preserves the disk intersection.
\uses{def:generators, ax:lens_preserved}
\end{lemma}

\begin{lemma}[GenA Isometric]\label{lem:genA_isometric}
\lean{genA_isometric_on_intersection}
\leanok
The generator $A$ is an isometry on the disk intersection.
\uses{def:generators}
\end{lemma}

\begin{lemma}[GenB Isometric]\label{lem:genB_isometric}
\lean{genB_isometric_on_intersection}
\leanok
The generator $B$ is an isometry on the disk intersection.
\uses{def:generators}
\end{lemma}

\section{Segment Maps}

\begin{definition}[Segment Map 1]\label{def:map1}
\lean{map1}
\leanok
The first segment map constructed from compositions of generators.
\uses{def:generators}
\end{definition}

\begin{definition}[Segment Map 2]\label{def:map2}
\lean{map2}
\leanok
The second segment map constructed from compositions of generators.
\uses{def:generators}
\end{definition}

\begin{definition}[Segment Map 3]\label{def:map3}
\lean{map3}
\leanok
The third segment map constructed from compositions of generators.
\uses{def:generators}
\end{definition}

\begin{axiom}[Map1 Disk Membership 1]\label{ax:map1_z1}
\lean{map1_new_z1_in_left_disk}
A specific computed point under map1 lies in the left disk.
\uses{def:map1, def:r_crit}
\end{axiom}

\begin{axiom}[Map1 Disk Membership 2]\label{ax:map1_z2}
\lean{map1_new_z2_in_right_disk}
A specific computed point under map1 lies in the right disk.
\uses{def:map1, def:r_crit}
\end{axiom}

\begin{axiom}[Map1 Disk Membership 3]\label{ax:map1_z3}
\lean{map1_new_z3_in_left_disk}
A specific computed point under map1 lies in the left disk.
\uses{def:map1, def:r_crit}
\end{axiom}

\begin{axiom}[Map1 Disk Membership 4]\label{ax:map1_z4}
\lean{map1_new_z4_in_right_disk}
A specific computed point under map1 lies in the right disk.
\uses{def:map1, def:r_crit}
\end{axiom}

\begin{theorem}[Map1 Endpoint E']\label{thm:map1_endpoint_Eprime}
\lean{map1_endpoint_E'}
\leanok
$\text{map1}(E') = G$.
\uses{def:map1, def:E, def:Eprime, def:G, ax:map1_z1, ax:map1_z2, ax:map1_z3, ax:map1_z4}
\end{theorem}

\begin{theorem}[Map1 Endpoint F']\label{thm:map1_endpoint_Fprime}
\lean{map1_endpoint_F'}
\leanok
$\text{map1}(F') = F$.
\uses{def:map1, def:F, ax:map1_z1, ax:map1_z2, ax:map1_z3, ax:map1_z4}
\end{theorem}

\begin{theorem}[Map2 Endpoint F']\label{thm:map2_endpoint_Fprime}
\lean{map2_sends_F'_to_F}
\leanok
$\text{map2}(F') = F$.
\uses{def:map2, def:F}
\end{theorem}

\begin{theorem}[Map2 Endpoint G']\label{thm:map2_endpoint_Gprime}
\lean{map2_sends_G'_to_E}
\leanok
$\text{map2}(G') = E$.
\uses{def:map2, def:G, def:E}
\end{theorem}

\begin{theorem}[Map3 Endpoint G']\label{thm:map3_endpoint_Gprime}
\lean{map3_sends_G'_to_E'}
\leanok
$\text{map3}(G') = E'$.
\uses{def:map3, def:G, def:Eprime}
\end{theorem}

\begin{theorem}[Map3 Endpoint E]\label{thm:map3_endpoint_E}
\lean{map3_sends_E_to_G}
\leanok
$\text{map3}(E) = G$.
\uses{def:map3, def:E, def:G}
\end{theorem}

\section{Interval Exchange Transformation}

\begin{definition}[IET Lengths]\label{def:iet_lengths}
\lean{length1, length2, length3}
\leanok
The three interval lengths for the induced interval exchange transformation.
\end{definition}

\begin{lemma}[Lengths Sum to One]\label{lem:lengths_sum}
\lean{lengths_sum_to_one}
\leanok
$\text{length}_1 + \text{length}_2 + \text{length}_3 = 1$.
\uses{def:iet_lengths}
\end{lemma}

\begin{lemma}[Length Golden Ratios]\label{lem:length_golden}
\lean{interval_lengths_golden_ratio_relations}
\leanok
$\text{length}_2 = \phi \cdot \text{length}_1$ and $\text{length}_3 = \phi \cdot \text{length}_2$.
\uses{def:iet_lengths}
\end{lemma}

\begin{definition}[GG5 Induced IET]\label{def:gg5_iet}
\lean{GG5_induced_IET}
\leanok
The interval exchange transformation induced by the three segment maps.
\uses{def:map1, def:map2, def:map3, def:iet_lengths}
\end{definition}

\begin{lemma}[IET Rotation Irrational]\label{lem:iet_rotation_irrational}
\lean{GG5_IET_rotation_irrational}
\leanok
The rotation ratio $\text{length}_2 / \text{length}_1$ is irrational.
\uses{def:gg5_iet, lem:length_golden}
\end{lemma}

\section{Orbit Theory}

\begin{definition}[Orbit Set]\label{def:orbit_set}
\lean{orbitSet}
\leanok
The orbit of a point under a map.
\end{definition}

\begin{definition}[Periodic]\label{def:periodic}
\lean{IsPeriodic}
\leanok
A point is periodic if it returns to itself after finitely many iterations.
\uses{def:orbit_set}
\end{definition}

\begin{lemma}[Finite Orbit Has Collision]\label{lem:finite_orbit_collision}
\lean{finite_orbit_has_collision}
\leanok
A finite orbit must have a collision.
\uses{def:orbit_set}
\end{lemma}

\begin{lemma}[Finite Orbit Implies Periodic]\label{lem:finite_orbit_periodic}
\lean{finite_orbit_implies_periodic}
\leanok
A finite orbit contains a periodic point.
\uses{def:orbit_set, def:periodic, lem:finite_orbit_collision}
\end{lemma}

\begin{axiom}[Keane's Theorem]\label{ax:keane}
\lean{IET_irrational_rotation_no_periodic_orbits}
For an interval exchange transformation with irrational rotation, no point has a periodic orbit.
\uses{def:periodic}
\end{axiom}

\begin{axiom}[IET Maps to Self]\label{ax:iet_maps_to_self}
\lean{IET_maps_to_self}
An IET maps the interval $[0,1)$ to itself.
\end{axiom}

\begin{lemma}[No Periodic Implies Infinite]\label{lem:no_periodic_infinite}
\lean{no_orbit_point_periodic_implies_infinite}
\leanok
If no point in an orbit is periodic, the orbit is infinite.
\uses{def:orbit_set, def:periodic, lem:finite_orbit_periodic}
\end{lemma}

\begin{theorem}[IET Has Infinite Orbit]\label{thm:iet_infinite_orbit}
\lean{GG5_IET_has_infinite_orbit}
\leanok
The GG5 induced IET has an infinite orbit.
\uses{def:gg5_iet, ax:keane, ax:iet_maps_to_self, lem:iet_rotation_irrational, lem:no_periodic_infinite}
\end{theorem}

\section{Axiom Summary}

The formalization uses 7 axioms:

\textbf{Computational Axioms (5):}
\begin{itemize}
\item \ref{ax:map1_z1}, \ref{ax:map1_z2}, \ref{ax:map1_z3}, \ref{ax:map1_z4}: Disk membership assertions for specific computed points
\item \ref{ax:lens_preserved}: Lens intersection preservation under rotation
\end{itemize}

\textbf{Ergodic Theory Axioms (2):}
\begin{itemize}
\item \ref{ax:keane}: Keane's theorem (1975) - IETs with irrational rotation have no periodic orbits
\item \ref{ax:iet_maps_to_self}: IET domain preservation
\end{itemize}

All axioms are mathematically justified. The computational axioms could be proven via norm calculations (estimated 240-370 lines). Keane's theorem is a deep result in ergodic theory requiring substantial infrastructure (estimated 800-1200 lines).

